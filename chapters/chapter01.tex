\documentclass[../thesis.tex]{subfiles}

\begin{document}

Giáo dục Việt Nam Cộng hòa là nền giáo dục tại miền Nam Việt Nam dưới chính thể Việt Nam Cộng hòa.
Hiến pháp Việt Nam Cộng hòa quy định quyền tự do giáo dục, và quy định nền giáo dục cơ bản có tính cách bắt buộc và miễn phí, "nền giáo dục đại học được tự trị", và "những người có khả năng mà không có phương tiện sẽ được nâng đỡ để theo đuổi học vấn". Hệ thống giáo dục Việt Nam Cộng hòa gồm tiểu học, trung học và đại học, cùng với một mạng lưới các cơ sở giáo dục công lập, dân lập và tư thục ở cả ba bậc học và hệ thống tổ chức quản trị từ trung ương tới địa phương.
Nhìn chung, mô hình giáo dục ở miền Nam Việt Nam trong những năm 1960-1970 có khuynh hướng xa dần ảnh hưởng của Pháp vốn chú trọng đào tạo một số ít phần tử trí thức và có khuynh hướng thiên về lý thuyết, để chuyển sang mô hình giáo dục đại chúng và thực tiễn.
Từ năm 1917, chính quyền thuộc địa Pháp ở Việt Nam đã có một hệ thống giáo dục thống nhất cho cả ba miền Nam, Trung, Bắc, và cả Lào cùng Cao Miên. Hệ thống giáo dục thời Pháp thuộc có ba bậc: tiểu học, trung học, và đại học. Chương trình học là chương trình của Pháp, với một chút sửa đổi nhỏ áp dụng cho các cơ sở giáo dục ở Việt Nam, dùng tiếng Pháp làm ngôn ngữ chính, tiếng Việt chỉ là ngôn ngữ phụ. Sau khi Việt Nam tuyên bố độc lập vào ngày 17 tháng 4 năm 1945, chương trình học của Việt Nam - còn gọi là chương trình Hoàng Xuân Hãn ban hành trong khoảng thời gian từ 20 tháng 4 đến 20 tháng 6 năm 1945 dưới thời chính phủ Trần Trọng Kim của Đế quốc Việt Nam - được đem ra áp dụng ở miền Trung và miền Bắc. Riêng ở miền Nam, vì có sự trở lại của người Pháp nên chương trình Pháp vẫn còn tiếp tục cho đến giữa thập niên 1950. Đến thời Đệ Nhất Cộng hòa thì chương trình Việt mới được áp dụng ở miền Nam để thay thế cho chương trình cũ của Pháp. Cũng từ đây, các nhà lãnh đạo giáo dục người Việt Nam mới có cơ hội đóng vai trò lãnh đạo.\footnote{Nguyễn Thanh Liêm (2006), tr. 19–21.}
Ngay từ những ngày đầu hình thành nền Đệ Nhất Cộng hòa, những người làm công tác giáo dục ở miền Nam đã bắt đầu xây dựng nền móng quan trọng cho nền giáo dục quốc gia, tìm câu trả lời cho những vấn đề giáo dục cốt yếu. Những vấn đề đó là: triết lý giáo dục, mục tiêu giáo dục, chương trình học, tài liệu giáo khoa và phương tiện học tập, vai trò của nhà giáo, cơ sở vật chất và trang thiết bị trường học, đánh giá kết quả học tập, và tổ chức quản trị.[4] Nhìn chung, người ta thấy mô hình giáo dục ở miền Nam Việt Nam trong những năm 1970 có khuynh hướng xa dần ảnh hưởng của Pháp vốn chú trọng đào tạo một số ít phần tử ưu tú trong xã hội và có khuynh hướng thiên về lý thuyết, để chấp nhận mô hình giáo dục Hoa Kỳ có tính cách đại chúng và thực tiễn.
Mặc dù tồn tại chỉ trong 20 năm, từ 1955 đến 1975, bị ảnh hưởng nặng nề bởi chiến tranh và những bất ổn chính trị thường xảy ra, phần thì ngân sách eo hẹp do phần lớn ngân sách quốc gia phải dành cho quốc phòng và nội vụ (trên 40\% ngân sách quốc gia dành cho quốc phòng, khoảng 13\% cho nội vụ, chỉ khoảng 7–7,5\% cho giáo dục), nhưng theo Nguyễn Thanh Liêm (cựu Thứ trưởng Bộ Quốc gia Giáo dục Việt Nam Cộng hòa), nền giáo dục đã phát triển nhanh, đáp ứng được nhu cầu gia tăng nhanh chóng của người dân, đào tạo được một lớp người có học vấn và có khả năng chuyên môn đóng góp vào việc xây dựng quốc gia và tạo được sự nghiệp ngay cả ở các quốc gia phát triển. Kết quả này là nhờ các nhà giáo có ý thức rõ ràng về sứ mạng giáo dục, có ý thức trách nhiệm và dành nhiều tâm huyết đóng góp cho nghề nghiệp, cũng như nhiều bậc phụ huynh đã đóng góp tài chính cho việc xây dựng nền giáo dục, và nhờ những nhà lãnh đạo giáo dục đã có những ý tưởng, sáng kiến, và nỗ lực mang lại sự tiến bộ cho nền giáo dục ở miền Nam Việt Nam.[3]

Năm học 1973–1974, Việt Nam Cộng hòa có một phần năm (20\%) dân số là học sinh và sinh viên đang đi học trong các cơ sở giáo dục. Con số này bao gồm 3.101.560 học sinh tiểu học, 1.091.779 học sinh trung học, và 101.454 sinh viên đại học[5] Đến năm 1975, tổng số sinh viên trong các viện đại học ở miền Nam là khoảng 150.000 người (không tính các sinh viên theo học ở Học viện Quốc gia Hành chánh và ở các trường đại học cộng đồng), chiếm 0,75\% dân số.[6]

Hệ thống giáo dục Việt Nam Cộng hòa thực hiện phổ cập tiểu học, mọi trẻ em đến 6 tuổi đều có quyền đăng ký đi học tại trường công. Tuy nhiên tới lớp 5 (lớp cuối cùng của cấp tiểu học), một loạt các kỳ thi được đề ra với tỷ lệ đánh trượt cao ở mọi giai đoạn:

Học hết tiểu học phải trải qua kỳ thi vào trung học đệ Nhất cấp. Kỳ thi có tính chọn lọc khá cao, tỷ lệ đậu vào trường công khoảng 62\%[7], số 38\% bị trượt phải vào trường tư và tự trang trải học phí.
Đến cuối năm lớp 11, học sinh phải thi Tú tài I, cuối năm lớp 12 phải thi Tú tài II. Nói chung Tú tài I tỷ lệ đậu chỉ khoảng 15–30\% và Tú tài II khoảng 30–45\%.[8] Nam giới thi hỏng Tú tài I phải trình diện nhập ngũ quân đội[9] và đi quân dịch hai năm[10] hoặc vào trường hạ sĩ quan Đồng Đế ở Nha Trang.
Do số thí sinh bị đánh trượt cao, học sinh thời Việt Nam Cộng hòa phải chịu áp lực rất lớn về thi cử nên phải học tập rất vất vả[11], và chỉ khoảng 24\% tổng số thiếu niên ở lứa tuổi từ 12 đến 18 là được đi học[12][13].
Tới năm 1974, tỷ lệ người dân biết đọc và viết của Việt Nam Cộng hòa ước tính vào khoảng 70\% dân số.[5]

\end{document}