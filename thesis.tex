\documentclass[a4paper,12pt,twoside]{report}
\usepackage[utf8]{vietnam}
\usepackage[top=3cm,inner=3.5cm,outer=2cm,bottom=3cm]{geometry}	% Dùng để set margin
\usepackage{graphicx}			% Dùng để chèn hình ảnh 
\graphicspath{ {images/} }		% Folder chứa hình ảnh
\usepackage{subfiles}			% Dùng để chia thành các files nhỏ
\usepackage{fancyhdr}			% Dùng để tạo header và footer
\usepackage{listings}			% Dùng để chèn source code vào luận văn

% Định dạng source code
\lstset{
	basicstyle=\small\ttfamily, 	% Font family và font size 
	numbers=left,					% Hiển thị số thứ tự dòng				
	tabsize=4,						% Tab size 
	frame=leftline,					% Hiển thị 1 gạch bên tay trái source code 
	xleftmargin=1cm					% Set margin-left 
}

\title{Nghiên cứu giáo dục Việt Nam Cộng Hoà}	% Tiêu đề luận văn
\author{Dương Lữ Điện}							% Người thực hiện 

% Header và footer 
\pagestyle{fancy}
\fancyhf{}
\rhead{Giảng viên hướng dẫn:\\TS. ABCXYZ}					% Header bên phải
\lhead{Đề tài:\\Nghiên cứu giáo dục Việt Nam Cộng Hoà}		% Header bên trái 
\rfoot{\thepage}					% Footer bên phải
\lfoot{Dương Lữ Điện - B1401133}	% Footer bên trái

\begin{document}

\subfile{bia-ngoai}		% Gọi file bìa ngoài
\subfile{bia-trong}		% Gọi file bìa trong

\pagenumbering{roman}	% Kiểu số trang: i, ii, iii, iv, v,...
\chapter*{NHẬN XÉT CỦA GIẢNG VIÊN}
\chapter*{LỜI CẢM ƠN}
\tableofcontents
\chapter{GIỚI THIỆU}
\pagenumbering{arabic}	% Kiểu số trang: 1, 2, 3,...
\subfile{chapters/chapter01}	% Gọi file chương 1, tương tự gọi các chương khác
\chapter{TRIẾT LÝ GIÁO DỤC}
\subfile{chapters/chapter02}	% Gọi file chương 2, tương tự gọi các chương khác

\end{document}